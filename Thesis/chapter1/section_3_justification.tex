\section{Justification}
\label{sec:justification}

A robust, modular long-term memory system is crucial for advancing state-of-the-art benchmarks in natural language long-input tasks, particularly QA and MHQA, which assess both language understanding and reasoning capabilities \cite{10.1561/1500000102}. These tasks are widely evaluated through benchmark datasets such as LoCoMo \cite{maharana-etal-2024-evaluating}, HotpotQA \cite{yang2018hotpotqa}, MuSiQue \cite{trivedi2021musique}, and 2WikiMultiHopQA \cite{ho-etal-2020-constructing}. \\

\noindent Beyond academic benchmarks, effective long-term memory architectures have significant real-world implications.  They could enhance QA systems, particularly in specialized domains such as law, where documents are lengthy and complex \cite{regnlp-ws-2025-1}. Additionally, they could improve personal companion and psychological counseling applications, where continuity and context retention are essential for meaningful interactions \cite{Zhong_Guo_Gao_Ye_Wang_2024}. \\ 

\noindent Moreover, memory is a fundamental component in the design of generalized language agents, as it enables more complex behavior. A clear example is human behavior simulations, where agents must rely on their memory to synthesize past experiences and reflect on previous interactions. The ability to retain and retrieve information allows agents to exhibit more coherent, evolving, and contextual behaviors, highlighting the importance of memory in agent-based systems \cite{10.1145/3586183.3606763}\cite{Hatalis_Christou_Myers_Jones_Lambert_Amos-Binks_Dannenhauer_Dannenhauer_2024}.  \\