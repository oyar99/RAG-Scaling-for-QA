\cleardoublepage
\chapter{Results}
\label{ch:results}
\label{ch:chapter4}

\section{Retrieval Results}

We first evaluated the retrieval performance of our baseline systems introduced in Section \ref{baselines_sec}. We found that ColBERTV2 consistently demonstrated strong performance across all datasets. In contrast, HippoRAG showed particularly high retrieval performance on MuSiQue, which is attributed to the dataset’s entity-centric design and also supported by the HippoRAG authors \cite{NEURIPS2024_6ddc001d}. However, HippoRAG underperformed on easier datasets relative to simpler methods like BM25. 

\noindent We also observe that MuSiQue and 2Wiki have the lowest overall retrieval scores. This is consistent with the design of these benchmarks, which introduce complexity through actual multi-hop reasoning.

\noindent Table \ref{tab:retrieval_results} summarizes retrieval performance across all systems, reporting \textbf{recall@k} for various values of $k$. Importantly, the vast majorify of questions in the datasets require no more than five passages as supporting evidence, which highlights the importance of high recall at low values of $k$ to reduce irrelevant content during the question answering phase.

\begin{table}[ht]
    \centering
    \small
    \resizebox{1.1\textwidth}{!}{
        \begin{tabular}{lccccccccccccccccccccccccc}
            \toprule
            & \multicolumn{5}{c}{\textbf{LoCoMo}} & \multicolumn{5}{c}{\textbf{HotpotQA}} & \multicolumn{5}{c}{\textbf{2Wiki}} & \multicolumn{5}{c}{\textbf{MuSiQue}} & \multicolumn{5}{c}{\textbf{Average}} \\
            \cmidrule(lr){2-6} \cmidrule(lr){7-11} \cmidrule(lr){12-16} \cmidrule(lr){17-21} \cmidrule(lr){22-26}
             & \scriptsize R@2 & \scriptsize R@5 & \scriptsize R@10 & \scriptsize R@20 & \scriptsize R@100 & \scriptsize R@2 & \scriptsize R@5 & \scriptsize R@10 & \scriptsize R@20 & \scriptsize R@100 & \scriptsize R@2 & \scriptsize R@5 & \scriptsize R@10 & \scriptsize R@20 & \scriptsize R@100 & \scriptsize R@2 & \scriptsize R@5 & \scriptsize R@10 & \scriptsize R@20 & \scriptsize R@100 & \scriptsize R@2 & \scriptsize R@5 & \scriptsize R@10 & \scriptsize R@20 & \scriptsize R@100 \\
             \midrule
             \scriptsize $\text{Default}^1$ & 94.4 & 99.5 & 99.9 & 100.0 & 100.0 & 100.0 & 100.0 & 100.0 & 100.0 & 100.0 & 89.1 & 100.0 & 100.0 & 100.0 & 100.0 & 81.1 & 100.0 & 100.0 & 100.0 & 100.0 & 91.2 & 99.9 & 100.0 & 100.0 & 100.0 \\
            \midrule
            \scriptsize BM25 & 41.6 & 52.7 & 60.3 & 66.3 & 78.6 & 50.7 & 63.6 & 74.6 & 
\underline{83.1} & \textbf{91.2} & 45.6 & 57.2 & 62.8 & 66.4 & 72.7 & 29.2 & 38.3 & 45.1 & 50.8 & 64.7 & 41.8 & 53.0 & 60.7 & 66.7 & 76.8 \\
            \scriptsize $\text{MSMarco-BERT}^2$ & \underline{42.5} & \underline{54.5} & \underline{62.2} & \underline{70.0} & \underline{86.0} & \underline{56.2} & 67.0 & 72.5 & 76.8 & 84.8 & \textbf{54.3} & \underline{62.7} & \underline{66.0} & \underline{68.4} & \underline{73.0} & \textbf{36.5} & \textbf{46.4} & \textbf{54.0} & \textbf{60.6} & \textbf{73.7} &
\underline{47.4} & \underline{57.7} & \underline{63.7} & \underline{68.9} & \underline{79.4} \\
            \scriptsize ColBERTV2 & \textbf{54.8} & \textbf{66.6} & \textbf{73.5} & \textbf{80.0} & \textbf{89.6} & \textbf{61.8} & \textbf{74.8} & \textbf{81.0} & \textbf{85.3} & \textbf{91.2} & \underline{53.6} & \textbf{62.8} & \textbf{67.1} & \textbf{69.9} & \textbf{74.5} & 35.1 & 45.4 & 52.1 & 57.7 & 71.4 & \textbf{51.3} & \textbf{62.4} & \textbf{68.4} & \textbf{73.3} & \textbf{81.7} \\
            %\scriptsize LLM (re-ranker)\cite{santhanam-etal-2022-colbertv2}\cite{sun-etal-2023-chatgpt} & - & - & - & - & - & - & - & - & - & - & - & - & - & - & - & - & - & - & - & - & - & - & - & - & - \\
            \midrule
            \scriptsize HippoRAG & - & - & - & - & - & 53.4 & \underline{67.8} & \underline{75.5} & 80.5 & \underline{88.5} & 39.6 & 48.3 & 54.0 & 59.0 & 68.2 & \underline{35.2} & \underline{46.1} & \underline{53.6} & \underline{60.1} & \underline{72.7} & - & - & - & - & - \\
            %\midrule
            %\scriptsize System I & - & - & - & - & - & - & - & -  & - & - & - & - & - & - & - & - & - & - & - & - & - & - & - & - & - \\
            %\scriptsize System II & - & - & - & - & - & - & - & -  & - & - & - & - & - & - & - & - & - & - & - & - & - & - & - & - & - \\
            %\scriptsize System III & - & - & - & - & - & - & - & - & - & - & - & - & - & - & - & - & - & - & - & - & - & - & - & - & - \\
            \bottomrule
            \multicolumn{16}{l}{\rule{0pt}{1.25em} $^1$ \footnotesize 
 Retrieved passages are always relevant.} \\
            \multicolumn{16}{l}{\rule{0pt}{0em} $^2$ \footnotesize 
 BERT embeddings trained on question answer pairs from the MS Marco dataset.}
        \end{tabular}
    }
    \caption{\small Retrieval performance.}
    \label{tab:retrieval_results}
\end{table}

\section{Question Answering Results}

\noindent Figure \ref{fig:scores_gpt4o} shows the results obtained using GPT-4o-mini across the four datasets using ColBERTV2, msmarco-bert-base-dot-v5, and BM25 retrievers. While performance generally improves with increasing $k$, the gains diminish as more irrelevant details begin to interfere with the model's ability to identify relevant information within its context.

\noindent Interestingly, a performance drop is observed on LoCoMo, HotpotQA, and 2Wiki under the full-context setup, which suggests that excessive context introduces noise. However, MuSiQue, the most complex dataset, benefits modestly from the added information.

\begin{figure}
\centering

\begin{subfigure}[b]{1.0\textwidth}
    % Define custom colors
\definecolor{emcolor}{rgb}{0.51,0.51,0.71}     % 130,130,180
\definecolor{r1color}{rgb}{0.39,0.75,0.51}     % 100,190,130
\definecolor{r2color}{rgb}{0.90,0.55,0.27}     % 230,140,70
\definecolor{l1color}{rgb}{0.68,0.85,0.90}     % 173,216,230

% Data tables
\pgfplotstableread[row sep=\\,col sep=&]{
K & EM & R1 & R2 & L1 \\
5   & 24.9 & 48.6 & 25.4 & 42.0 \\
10  & 25.4 & 50.0 & 25.8 & 42.0 \\
20  & 25.6 & 52.2 & 27.0 & 43.0 \\
100 & 26.6 & 53.0 & 27.0 & 45.0 \\
}\locomo

\pgfplotstableread[row sep=\\,col sep=&]{
K & EM & R1 & R2 & L1 \\
5   & 36.9 & 46.8 & 27.2 & 47.0 \\
10  & 39.8 & 50.5 & 29.2 & 51.0 \\
20  & 41.9 & 53.3 & 30.8 & 54.0 \\
100 & 42.5 & 55.2 & 31.2 & 56.0 \\
}\hotpotqa

\pgfplotstableread[row sep=\\,col sep=&]{
K & EM & R1 & R2 & L1 \\
5   & 27.2 & 31.3 & 23.5 & 28.0 \\
10  & 29.8 & 34.5 & 25.7 & 31.0 \\
20  & 33.1 & 38.3 & 28.6 & 35.0 \\
100 &  &  &  &  \\
}\twowiki

\pgfplotstableread[row sep=\\,col sep=&]{
K & EM & R1 & R2 & L1 \\
5   & 9.7 & 13.9 & 7.5 & 12.0 \\
10  & 11.7 & 16.6 & 9.0 & 14.0 \\
20  & 13.0 & 19.0 & 10.3 & 16.0 \\
100 & 12.5 & 19.7 & 10.3 & 16.0 \\
}\musique

% Plot
\begin{tikzpicture}
    \begin{groupplot}[
        group style={
            group size=4 by 1,
            horizontal sep=0.05cm,
            vertical sep=0cm,
        },
        width=0.35\textwidth,
        height=0.35\textwidth,
        xlabel={\tiny Docs},
        ylabel={\tiny Score},
        ymin=0, ymax=70,
        ytick={20,40,60},
        tick style={draw=none},
        legend style={
            at={(2,1.4)}, % centered above the plots
            anchor=north,
            legend columns=4,
            font=\tiny,
            draw=gray!40, % light gray border
            fill=white,   % optional: white background
            align={left}
        }, 
        legend cell align={left},
        axis line style={draw=none},
        grid=major,
        grid style={dotted},
        yticklabel style={font=\tiny},
        xticklabel style={font=\tiny},
        title style={font=\tiny},
        xtick=data,
        symbolic x coords={5,10,20,100},
        ylabel style={yshift=-2pt},
        xlabel style={yshift=2pt}
    ]

    % LoCoMo subplot
    \nextgroupplot[title={LoCoMo}]
    \addplot+[
        mark=triangle*,
        mark size=1.0pt,
        thick,
        dotted,
        draw=emcolor,
        mark options={fill=emcolor, draw=emcolor, solid}
    ] table[x=K, y=EM] {\locomo}; \addlegendentry{EM}

    \addplot+[
        mark=triangle*,
        mark size=1.0pt,
        thick,
        dotted,
        draw=r1color,
        mark options={fill=r1color, draw=r1color, solid}
    ] table[x=K, y=R1] {\locomo}; \addlegendentry{R1}

    \addplot+[
        mark=triangle*,
        mark size=1.0pt,
        thick,
        dotted,
        draw=r2color,
        mark options={fill=r2color, draw=r2color, solid}
    ] table[x=K, y=R2] {\locomo}; \addlegendentry{R2}

    \addplot+[
        mark=triangle*,
        mark size=1.0pt,
        thick,
        dotted,
        draw=l1color,
        mark options={fill=l1color, draw=l1color, solid}
    ] table[x=K, y=L1] {\locomo}; \addlegendentry{L1}

    % HotpotQA subplot
    \nextgroupplot[
        title={HotpotQA},
        ylabel={},
        yticklabels={},
        ytick style={draw=none}
    ]
    \addplot+[
        mark=triangle*,
        mark size=1.0pt,
        thick,
        dotted,
        draw=emcolor,
        mark options={fill=emcolor, draw=emcolor, solid}
    ] table[x=K, y=EM] {\hotpotqa};

    \addplot+[
        mark=triangle*,
        mark size=1.0pt,
        thick,
        dotted,
        draw=r1color,
        mark options={fill=r1color, draw=r1color, solid}
    ] table[x=K, y=R1] {\hotpotqa};

    \addplot+[
        mark=triangle*,
        mark size=1.0pt,
        thick,
        dotted,
        draw=r2color,
        mark options={fill=r2color, draw=r2color, solid}
    ] table[x=K, y=R2] {\hotpotqa};

    \addplot+[
        mark=triangle*,
        mark size=1.0pt,
        thick,
        dotted,
        draw=l1color,
        mark options={fill=l1color, draw=l1color, solid}
    ] table[x=K, y=L1] {\hotpotqa};

    % TwoWiki subplot
    \nextgroupplot[
        title={2Wiki},
        ylabel={},
        yticklabels={},
        ytick style={draw=none}
    ]
    \addplot+[
        mark=triangle*,
        mark size=1.0pt,
        thick,
        dotted,
        draw=emcolor,
        mark options={fill=emcolor, draw=emcolor, solid}
    ] table[x=K, y=EM] {\twowiki};

    \addplot+[
        mark=triangle*,
        mark size=1.0pt,
        thick,
        dotted,
        draw=r1color,
        mark options={fill=r1color, draw=r1color, solid}
    ] table[x=K, y=R1] {\twowiki};

    \addplot+[
        mark=triangle*,
        mark size=1.0pt,
        thick,
        dotted,
        draw=r2color,
        mark options={fill=r2color, draw=r2color, solid}
    ] table[x=K, y=R2] {\twowiki};

    \addplot+[
        mark=triangle*,
        mark size=1.0pt,
        thick,
        dotted,
        draw=l1color,
        mark options={fill=l1color, draw=l1color, solid}
    ] table[x=K, y=L1] {\twowiki};

    % MuSiQue subplot
    \nextgroupplot[
        title={MuSiQue},
        ylabel={},
        yticklabels={},
        ytick style={draw=none}
    ]
    \addplot+[
        mark=triangle*,
        mark size=1.0pt,
        thick,
        dotted,
        draw=emcolor,
        mark options={fill=emcolor, draw=emcolor, solid}
    ] table[x=K, y=EM] {\musique};

    \addplot+[
        mark=triangle*,
        mark size=1.0pt,
        thick,
        dotted,
        draw=r1color,
        mark options={fill=r1color, draw=r1color, solid}
    ] table[x=K, y=R1] {\musique};

    \addplot+[
        mark=triangle*,
        mark size=1.0pt,
        thick,
        dotted,
        draw=r2color,
        mark options={fill=r2color, draw=r2color, solid}
    ] table[x=K, y=R2] {\musique};

    \addplot+[
        mark=triangle*,
        mark size=1.0pt,
        thick,
        dotted,
        draw=l1color,
        mark options={fill=l1color, draw=l1color, solid}
    ] table[x=K, y=L1] {\musique};

    \end{groupplot}
\end{tikzpicture}
    \vspace{-1.0em}
    \caption{\scriptsize ColBERTV2}\label{fig:scores_colbert}
\end{subfigure}

\vspace{1.0em}

\begin{subfigure}[b]{0.49\textwidth}
    % Define custom colors
\definecolor{emcolor}{rgb}{0.51,0.51,0.71}     % 130,130,180
\definecolor{r1color}{rgb}{0.39,0.75,0.51}     % 100,190,130
\definecolor{r2color}{rgb}{0.90,0.55,0.27}     % 230,140,70
\definecolor{l1color}{rgb}{0.68,0.85,0.90}     % 173,216,230

% Data tables
\pgfplotstableread[row sep=\\,col sep=&]{
K & EM & R1 & R2 & L1 \\
5   & 21.5 & 42.0 & 21.0 & 34.0 \\
10  & 22.4 & 44.7 & 21.8 & 37.0 \\
20  & 24.6 & 48.4 & 24.5 & 41.0 \\
100 & 24.8 & 50.0 & 25.2 & 41.0 \\
}\locomo

\pgfplotstableread[row sep=\\,col sep=&]{
K & EM & R1 & R2 & L1 \\
5   & 32.2 & 40.5 & 22.7 & 41.0 \\
10  & 34.8 & 44.1 & 24.8 & 44.0 \\
20  & 36.7 & 46.6 & 26.2 & 47.0 \\
100 & 39.2 & 50.7 & 28.3 & 52.0  \\
}\hotpotqa

\pgfplotstableread[row sep=\\,col sep=&]{
K & EM & R1 & R2 & L1 \\
5   & 29.9 & 33.9 & 25.5 & 31.0 \\
10  & 32.4 & 37.0 & 27.9 & 34.0 \\
20  & 34.0 & 38.9 & 29.2 & 36.0 \\
100 &  &  &  &  \\
}\twowiki

\pgfplotstableread[row sep=\\,col sep=&]{
K & EM & R1 & R2 & L1 \\
5   & 8.4 & 12.6 & 7.2 & 10.0 \\
10  & 12.2 & 17.2 & 9.6 & 14.0 \\
20  & 13.9 & 20.4 & 11.2 & 17.0 \\
100 & 14.2 & 21.8 & 11.8 & 18.0 \\
}\musique

\begin{tikzpicture}
    \begin{groupplot}[
        group style={
            group size=4 by 1,
            horizontal sep=0.05cm,
            vertical sep=0cm,
        },
        width=0.45\linewidth,
        height=0.45\linewidth,
        xlabel={\tiny Docs},
        ylabel={Score},
        ymin=0, ymax=70,
        ytick={20,40,60},
        tick style={draw=none},
        axis line style={draw=none},
        grid=major,
        grid style={dotted},
        yticklabel style={font=\fontsize{1pt}{2.5pt}\selectfont},
        xticklabel style={font=\fontsize{1pt}{2.5pt}\selectfont, rotate=45, anchor=east},
        title style={font=\tiny},
        xtick=data,
        symbolic x coords={5,10,20,100},
        ylabel style={yshift=-20pt, font=\fontsize{1pt}{2.5pt}\selectfont},
        xlabel style={yshift=6pt, font=\fontsize{1pt}{2.5pt}\selectfont}
    ]

    % LoCoMo subplot
    \nextgroupplot[title={LoCoMo}]
    \addplot+[
        mark=triangle*,
        mark size=0.4pt,
        thick,
        dash pattern=on 1pt off 1pt,
        draw=emcolor,
        mark options={fill=emcolor, draw=emcolor, solid}
    ] table[x=K, y=EM] {\locomo};

    \addplot+[
        mark=triangle*,
        mark size=0.4pt,
        thick,
        dash pattern=on 1pt off 1pt,
        draw=r1color,
        mark options={fill=r1color, draw=r1color, solid}
    ] table[x=K, y=R1] {\locomo};

    \addplot+[
        mark=triangle*,
        mark size=0.4pt,
        thick,
        dash pattern=on 1pt off 1pt,
        draw=r2color,
        mark options={fill=r2color, draw=r2color, solid}
    ] table[x=K, y=R2] {\locomo};

    \addplot+[
        mark=triangle*,
        mark size=0.4pt,
        thick,
        dash pattern=on 1pt off 1pt,
        draw=l1color,
        mark options={fill=l1color, draw=l1color, solid}
    ] table[x=K, y=L1] {\locomo};

    % HotpotQA subplot
    \nextgroupplot[
        title={HotpotQA},
        ylabel={},
        yticklabels={},
        ytick style={draw=none}
    ]
    \addplot+[
        mark=triangle*,
        mark size=0.4pt,
        thick,
        dash pattern=on 1pt off 1pt,
        draw=emcolor,
        mark options={fill=emcolor, draw=emcolor, solid}
    ] table[x=K, y=EM] {\hotpotqa};

    \addplot+[
        mark=triangle*,
        mark size=0.4pt,
        thick,
        dash pattern=on 1pt off 1pt,
        draw=r1color,
        mark options={fill=r1color, draw=r1color, solid}
    ] table[x=K, y=R1] {\hotpotqa};

    \addplot+[
        mark=triangle*,
        mark size=0.4pt,
        thick,
        dash pattern=on 1pt off 1pt,
        draw=r2color,
        mark options={fill=r2color, draw=r2color, solid}
    ] table[x=K, y=R2] {\hotpotqa};

    \addplot+[
        mark=triangle*,
        mark size=0.4pt,
        thick,
        dash pattern=on 1pt off 1pt,
        draw=l1color,
        mark options={fill=l1color, draw=l1color, solid}
    ] table[x=K, y=L1] {\hotpotqa};

    % TwoWiki subplot
    \nextgroupplot[
        title={2Wiki},
        ylabel={},
        yticklabels={},
        ytick style={draw=none}
    ]
    \addplot+[
        mark=triangle*,
        mark size=0.4pt,
        thick,
        dash pattern=on 1pt off 1pt,
        draw=emcolor,
        mark options={fill=emcolor, draw=emcolor, solid}
    ] table[x=K, y=EM] {\twowiki};

    \addplot+[
        mark=triangle*,
        mark size=0.4pt,
        thick,
        dash pattern=on 1pt off 1pt,
        draw=r1color,
        mark options={fill=r1color, draw=r1color, solid}
    ] table[x=K, y=R1] {\twowiki};

    \addplot+[
        mark=triangle*,
        mark size=0.4pt,
        thick,
        dash pattern=on 1pt off 1pt,
        draw=r2color,
        mark options={fill=r2color, draw=r2color, solid}
    ] table[x=K, y=R2] {\twowiki};

    \addplot+[
        mark=triangle*,
        mark size=0.4pt,
        thick,
        dash pattern=on 1pt off 1pt,
        draw=l1color,
        mark options={fill=l1color, draw=l1color, solid}
    ] table[x=K, y=L1] {\twowiki};

    % MuSiQue subplot
    \nextgroupplot[
        title={MuSiQue},
        ylabel={},
        yticklabels={},
        ytick style={draw=none}
    ]
    \addplot+[
        mark=triangle*,
        mark size=0.4pt,
        thick,
        dash pattern=on 1pt off 1pt,
        draw=emcolor,
        mark options={fill=emcolor, draw=emcolor, solid}
    ] table[x=K, y=EM] {\musique};

    \addplot+[
        mark=triangle*,
        mark size=0.4pt,
        thick,
        dash pattern=on 1pt off 1pt,
        draw=r1color,
        mark options={fill=r1color, draw=r1color, solid}
    ] table[x=K, y=R1] {\musique};

    \addplot+[
        mark=triangle*,
        mark size=0.4pt,
        thick,
        dash pattern=on 1pt off 1pt,
        draw=r2color,
        mark options={fill=r2color, draw=r2color, solid}
    ] table[x=K, y=R2] {\musique};

    \addplot+[
        mark=triangle*,
        mark size=0.4pt,
        thick,
        dash pattern=on 1pt off 1pt,
        draw=l1color,
        mark options={fill=l1color, draw=l1color, solid}
    ] table[x=K, y=L1] {\musique};

    \end{groupplot}
\end{tikzpicture}
    \caption{\scriptsize MSMarco-BERT}\label{fig:scores_dense}
\end{subfigure}
\begin{subfigure}[b]{0.49\textwidth}
    % Define custom colors
\definecolor{emcolor}{rgb}{0.51,0.51,0.71}     % 130,130,180
\definecolor{r1color}{rgb}{0.39,0.75,0.51}     % 100,190,130
\definecolor{r2color}{rgb}{0.90,0.55,0.27}     % 230,140,70
\definecolor{l1color}{rgb}{0.68,0.85,0.90}     % 173,216,230

% Data tables
\pgfplotstableread[row sep=\\,col sep=&]{
K & EM & R1 & R2 & L1 \\
5   & 23.3 & 43.9 & 23.6 & 39.0 \\
10  & 24.8 & 47.1 & 25.6 & 42.0 \\
20  & 25.4 & 49.2 & 26.0 & 45.0 \\
100 & 27.5 & 53.4 & 29.2 & 51.0 \\
}\locomo

\pgfplotstableread[row sep=\\,col sep=&]{
K & EM & R1 & R2 & L1 \\
5   & 36.2 & 44.6 & 25.1 & 45.0 \\
10  & 41.2 & 50.7 & 28.7 & 51.0 \\
20  & 45.1 & 55.3 & 31.3 & 56.0 \\
100 & 46.1 & 57.6 & 31.8 &  \\
}\hotpotqa

\pgfplotstableread[row sep=\\,col sep=&]{
K & EM & R1 & R2 & L1 \\
5   & 26.3 & 28.8 & 21.6 & 26.0 \\
10  & 29.8 & 32.6 & 24.2 & 30.0 \\
20  & 30.8 & 34.0 & 25.1 & 31.0 \\
100 & 33.0 & 36.7 & 26.3 &  \\
}\twowiki

\pgfplotstableread[row sep=\\,col sep=&]{
K & EM & R1 & R2 & L1 \\
5   & 7.6 & 10.3 & 5.6 & 9.0 \\
10  & 9.0 & 12.8 & 6.9 & 11.0 \\
20  & 10.9 & 15.1 & 8.3 & 13.0 \\
100 & 12.4 & 17.1 & 9.9 & 15.0 \\
}\musique

\begin{tikzpicture}
    \begin{groupplot}[
        group style={
            group size=4 by 1,
            horizontal sep=0.05cm,
            vertical sep=0cm,
        },
        width=0.45\linewidth,
        height=0.45\linewidth,
        xlabel={\tiny Docs},
        ylabel={Score},
        ymin=0, ymax=70,
        ytick={20,40,60},
        tick style={draw=none},
        axis line style={draw=none},
        grid=major,
        grid style={dotted},
        yticklabel style={font=\fontsize{1pt}{2.5pt}\selectfont},
        xticklabel style={font=\fontsize{1pt}{2.5pt}\selectfont, rotate=45, anchor=east},
        title style={font=\tiny},
        xtick=data,
        symbolic x coords={5,10,20,100},
        ylabel style={yshift=-20pt, font=\fontsize{1pt}{2.5pt}\selectfont},
        xlabel style={yshift=6pt, font=\fontsize{1pt}{2.5pt}\selectfont}
    ]

    % LoCoMo subplot
    \nextgroupplot[title={LoCoMo}]
    \addplot+[
        mark=triangle*,
        mark size=0.4pt,
        thick,
        dash pattern=on 1pt off 1pt,
        draw=emcolor,
        mark options={fill=emcolor, draw=emcolor, solid}
    ] table[x=K, y=EM] {\locomo};

    \addplot+[
        mark=triangle*,
        mark size=0.4pt,
        thick,
        dash pattern=on 1pt off 1pt,
        draw=r1color,
        mark options={fill=r1color, draw=r1color, solid}
    ] table[x=K, y=R1] {\locomo};

    \addplot+[
        mark=triangle*,
        mark size=0.4pt,
        thick,
        dash pattern=on 1pt off 1pt,
        draw=r2color,
        mark options={fill=r2color, draw=r2color, solid}
    ] table[x=K, y=R2] {\locomo};

    \addplot+[
        mark=triangle*,
        mark size=0.4pt,
        thick,
        dash pattern=on 1pt off 1pt,
        draw=l1color,
        mark options={fill=l1color, draw=l1color, solid}
    ] table[x=K, y=L1] {\locomo};

    % HotpotQA subplot
    \nextgroupplot[
        title={HotpotQA},
        ylabel={},
        yticklabels={},
        ytick style={draw=none}
    ]
    \addplot+[
        mark=triangle*,
        mark size=0.4pt,
        thick,
        dash pattern=on 1pt off 1pt,
        draw=emcolor,
        mark options={fill=emcolor, draw=emcolor, solid}
    ] table[x=K, y=EM] {\hotpotqa};

    \addplot+[
        mark=triangle*,
        mark size=0.4pt,
        thick,
        dash pattern=on 1pt off 1pt,
        draw=r1color,
        mark options={fill=r1color, draw=r1color, solid}
    ] table[x=K, y=R1] {\hotpotqa};

    \addplot+[
        mark=triangle*,
        mark size=0.4pt,
        thick,
        dash pattern=on 1pt off 1pt,
        draw=r2color,
        mark options={fill=r2color, draw=r2color, solid}
    ] table[x=K, y=R2] {\hotpotqa};

    \addplot+[
        mark=triangle*,
        mark size=0.4pt,
        thick,
        dash pattern=on 1pt off 1pt,
        draw=l1color,
        mark options={fill=l1color, draw=l1color, solid}
    ] table[x=K, y=L1] {\hotpotqa};

    % TwoWiki subplot
    \nextgroupplot[
        title={2Wiki},
        ylabel={},
        yticklabels={},
        ytick style={draw=none}
    ]
    \addplot+[
        mark=triangle*,
        mark size=0.4pt,
        thick,
        dash pattern=on 1pt off 1pt,
        draw=emcolor,
        mark options={fill=emcolor, draw=emcolor, solid}
    ] table[x=K, y=EM] {\twowiki};

    \addplot+[
        mark=triangle*,
        mark size=0.4pt,
        thick,
        dash pattern=on 1pt off 1pt,
        draw=r1color,
        mark options={fill=r1color, draw=r1color, solid}
    ] table[x=K, y=R1] {\twowiki};

    \addplot+[
        mark=triangle*,
        mark size=0.4pt,
        thick,
        dash pattern=on 1pt off 1pt,
        draw=r2color,
        mark options={fill=r2color, draw=r2color, solid}
    ] table[x=K, y=R2] {\twowiki};

    \addplot+[
        mark=triangle*,
        mark size=0.4pt,
        thick,
        dash pattern=on 1pt off 1pt,
        draw=l1color,
        mark options={fill=l1color, draw=l1color, solid}
    ] table[x=K, y=L1] {\twowiki};

    % MuSiQue subplot
    \nextgroupplot[
        title={MuSiQue},
        ylabel={},
        yticklabels={},
        ytick style={draw=none}
    ]
    \addplot+[
        mark=triangle*,
        mark size=0.4pt,
        thick,
        dash pattern=on 1pt off 1pt,
        draw=emcolor,
        mark options={fill=emcolor, draw=emcolor, solid}
    ] table[x=K, y=EM] {\musique};

    \addplot+[
        mark=triangle*,
        mark size=0.4pt,
        thick,
        dash pattern=on 1pt off 1pt,
        draw=r1color,
        mark options={fill=r1color, draw=r1color, solid}
    ] table[x=K, y=R1] {\musique};

    \addplot+[
        mark=triangle*,
        mark size=0.4pt,
        thick,
        dash pattern=on 1pt off 1pt,
        draw=r2color,
        mark options={fill=r2color, draw=r2color, solid}
    ] table[x=K, y=R2] {\musique};

    \addplot+[
        mark=triangle*,
        mark size=0.4pt,
        thick,
        dash pattern=on 1pt off 1pt,
        draw=l1color,
        mark options={fill=l1color, draw=l1color, solid}
    ] table[x=K, y=L1] {\musique};

    \end{groupplot}
\end{tikzpicture}
    \caption{\scriptsize BM25}\label{fig:scores_bm25}
\end{subfigure}

\caption{Performance results of GPT-4o-mini with different retrievers.}
\label{fig:scores_gpt4o}
\end{figure}

\noindent We also evaluate Qwen2.5-14B using a slightly adapted prompt, see Appendix \ref{fig:qa-base-qwen}.  As shown in Figure \ref{fig:scores_qwen}, its performance closely matches that of GPT-4o-mini, demonstrating that smaller models with strong instruction tuning can achieve competitive results.

\begin{figure}
\centering

\begin{subfigure}[b]{1.0\textwidth}
    % Define custom colors
\definecolor{emcolor}{rgb}{0.51,0.51,0.71}     % 130,130,180
\definecolor{r1color}{rgb}{0.39,0.75,0.51}     % 100,190,130
\definecolor{r2color}{rgb}{0.90,0.55,0.27}     % 230,140,70
\definecolor{l1color}{rgb}{0.68,0.85,0.90}     % 173,216,230

% Data tables
\pgfplotstableread[row sep=\\,col sep=&]{
K & EM & R1 & R2 & L1 \\
5   & 24.9 & 48.6 & 25.4 & 42.0 \\
10  & 25.4 & 50.0 & 25.8 & 42.0 \\
20  & 25.6 & 52.2 & 27.0 & 43.0 \\
100 & 26.6 & 53.0 & 27.0 & 45.0 \\
}\locomo

\pgfplotstableread[row sep=\\,col sep=&]{
K & EM & R1 & R2 & L1 \\
5   & 36.9 & 46.8 & 27.2 & 47.0 \\
10  & 39.8 & 50.5 & 29.2 & 51.0 \\
20  & 41.9 & 53.3 & 30.8 & 54.0 \\
100 & 42.5 & 55.2 & 31.2 & 56.0 \\
}\hotpotqa

\pgfplotstableread[row sep=\\,col sep=&]{
K & EM & R1 & R2 & L1 \\
5   & 27.2 & 31.3 & 23.5 & 28.0 \\
10  & 29.8 & 34.5 & 25.7 & 31.0 \\
20  & 33.1 & 38.3 & 28.6 & 35.0 \\
100 &  &  &  &  \\
}\twowiki

\pgfplotstableread[row sep=\\,col sep=&]{
K & EM & R1 & R2 & L1 \\
5   & 9.7 & 13.9 & 7.5 & 12.0 \\
10  & 11.7 & 16.6 & 9.0 & 14.0 \\
20  & 13.0 & 19.0 & 10.3 & 16.0 \\
100 & 12.5 & 19.7 & 10.3 & 16.0 \\
}\musique

% Plot
\begin{tikzpicture}
    \begin{groupplot}[
        group style={
            group size=4 by 1,
            horizontal sep=0.05cm,
            vertical sep=0cm,
        },
        width=0.35\textwidth,
        height=0.35\textwidth,
        xlabel={\tiny Docs},
        ylabel={\tiny Score},
        ymin=0, ymax=70,
        ytick={20,40,60},
        tick style={draw=none},
        legend style={
            at={(2,1.4)}, % centered above the plots
            anchor=north,
            legend columns=4,
            font=\tiny,
            draw=gray!40, % light gray border
            fill=white,   % optional: white background
            align={left}
        }, 
        legend cell align={left},
        axis line style={draw=none},
        grid=major,
        grid style={dotted},
        yticklabel style={font=\tiny},
        xticklabel style={font=\tiny},
        title style={font=\tiny},
        xtick=data,
        symbolic x coords={5,10,20,100},
        ylabel style={yshift=-2pt},
        xlabel style={yshift=2pt}
    ]

    % LoCoMo subplot
    \nextgroupplot[title={LoCoMo}]
    \addplot+[
        mark=triangle*,
        mark size=1.0pt,
        thick,
        dotted,
        draw=emcolor,
        mark options={fill=emcolor, draw=emcolor, solid}
    ] table[x=K, y=EM] {\locomo}; \addlegendentry{EM}

    \addplot+[
        mark=triangle*,
        mark size=1.0pt,
        thick,
        dotted,
        draw=r1color,
        mark options={fill=r1color, draw=r1color, solid}
    ] table[x=K, y=R1] {\locomo}; \addlegendentry{R1}

    \addplot+[
        mark=triangle*,
        mark size=1.0pt,
        thick,
        dotted,
        draw=r2color,
        mark options={fill=r2color, draw=r2color, solid}
    ] table[x=K, y=R2] {\locomo}; \addlegendentry{R2}

    \addplot+[
        mark=triangle*,
        mark size=1.0pt,
        thick,
        dotted,
        draw=l1color,
        mark options={fill=l1color, draw=l1color, solid}
    ] table[x=K, y=L1] {\locomo}; \addlegendentry{L1}

    % HotpotQA subplot
    \nextgroupplot[
        title={HotpotQA},
        ylabel={},
        yticklabels={},
        ytick style={draw=none}
    ]
    \addplot+[
        mark=triangle*,
        mark size=1.0pt,
        thick,
        dotted,
        draw=emcolor,
        mark options={fill=emcolor, draw=emcolor, solid}
    ] table[x=K, y=EM] {\hotpotqa};

    \addplot+[
        mark=triangle*,
        mark size=1.0pt,
        thick,
        dotted,
        draw=r1color,
        mark options={fill=r1color, draw=r1color, solid}
    ] table[x=K, y=R1] {\hotpotqa};

    \addplot+[
        mark=triangle*,
        mark size=1.0pt,
        thick,
        dotted,
        draw=r2color,
        mark options={fill=r2color, draw=r2color, solid}
    ] table[x=K, y=R2] {\hotpotqa};

    \addplot+[
        mark=triangle*,
        mark size=1.0pt,
        thick,
        dotted,
        draw=l1color,
        mark options={fill=l1color, draw=l1color, solid}
    ] table[x=K, y=L1] {\hotpotqa};

    % TwoWiki subplot
    \nextgroupplot[
        title={2Wiki},
        ylabel={},
        yticklabels={},
        ytick style={draw=none}
    ]
    \addplot+[
        mark=triangle*,
        mark size=1.0pt,
        thick,
        dotted,
        draw=emcolor,
        mark options={fill=emcolor, draw=emcolor, solid}
    ] table[x=K, y=EM] {\twowiki};

    \addplot+[
        mark=triangle*,
        mark size=1.0pt,
        thick,
        dotted,
        draw=r1color,
        mark options={fill=r1color, draw=r1color, solid}
    ] table[x=K, y=R1] {\twowiki};

    \addplot+[
        mark=triangle*,
        mark size=1.0pt,
        thick,
        dotted,
        draw=r2color,
        mark options={fill=r2color, draw=r2color, solid}
    ] table[x=K, y=R2] {\twowiki};

    \addplot+[
        mark=triangle*,
        mark size=1.0pt,
        thick,
        dotted,
        draw=l1color,
        mark options={fill=l1color, draw=l1color, solid}
    ] table[x=K, y=L1] {\twowiki};

    % MuSiQue subplot
    \nextgroupplot[
        title={MuSiQue},
        ylabel={},
        yticklabels={},
        ytick style={draw=none}
    ]
    \addplot+[
        mark=triangle*,
        mark size=1.0pt,
        thick,
        dotted,
        draw=emcolor,
        mark options={fill=emcolor, draw=emcolor, solid}
    ] table[x=K, y=EM] {\musique};

    \addplot+[
        mark=triangle*,
        mark size=1.0pt,
        thick,
        dotted,
        draw=r1color,
        mark options={fill=r1color, draw=r1color, solid}
    ] table[x=K, y=R1] {\musique};

    \addplot+[
        mark=triangle*,
        mark size=1.0pt,
        thick,
        dotted,
        draw=r2color,
        mark options={fill=r2color, draw=r2color, solid}
    ] table[x=K, y=R2] {\musique};

    \addplot+[
        mark=triangle*,
        mark size=1.0pt,
        thick,
        dotted,
        draw=l1color,
        mark options={fill=l1color, draw=l1color, solid}
    ] table[x=K, y=L1] {\musique};

    \end{groupplot}
\end{tikzpicture}
    \vspace{-1.0em}
    \caption{\scriptsize ColBERTV2}\label{fig:scores_colbert_qwen}
\end{subfigure}

\vspace{1.0em}

\begin{subfigure}[b]{0.49\textwidth}
    % Define custom colors
\definecolor{emcolor}{rgb}{0.51,0.51,0.71}     % 130,130,180
\definecolor{r1color}{rgb}{0.39,0.75,0.51}     % 100,190,130
\definecolor{r2color}{rgb}{0.90,0.55,0.27}     % 230,140,70
\definecolor{l1color}{rgb}{0.68,0.85,0.90}     % 173,216,230

% Data tables
\pgfplotstableread[row sep=\\,col sep=&]{
K & EM & R1 & R2 & L1 \\
5   & 21.5 & 42.0 & 21.0 & 34.0 \\
10  & 22.4 & 44.7 & 21.8 & 37.0 \\
20  & 24.6 & 48.4 & 24.5 & 41.0 \\
100 & 24.8 & 50.0 & 25.2 & 41.0 \\
}\locomo

\pgfplotstableread[row sep=\\,col sep=&]{
K & EM & R1 & R2 & L1 \\
5   & 32.2 & 40.5 & 22.7 & 41.0 \\
10  & 34.8 & 44.1 & 24.8 & 44.0 \\
20  & 36.7 & 46.6 & 26.2 & 47.0 \\
100 & 39.2 & 50.7 & 28.3 & 52.0  \\
}\hotpotqa

\pgfplotstableread[row sep=\\,col sep=&]{
K & EM & R1 & R2 & L1 \\
5   & 29.9 & 33.9 & 25.5 & 31.0 \\
10  & 32.4 & 37.0 & 27.9 & 34.0 \\
20  & 34.0 & 38.9 & 29.2 & 36.0 \\
100 &  &  &  &  \\
}\twowiki

\pgfplotstableread[row sep=\\,col sep=&]{
K & EM & R1 & R2 & L1 \\
5   & 8.4 & 12.6 & 7.2 & 10.0 \\
10  & 12.2 & 17.2 & 9.6 & 14.0 \\
20  & 13.9 & 20.4 & 11.2 & 17.0 \\
100 & 14.2 & 21.8 & 11.8 & 18.0 \\
}\musique

\begin{tikzpicture}
    \begin{groupplot}[
        group style={
            group size=4 by 1,
            horizontal sep=0.05cm,
            vertical sep=0cm,
        },
        width=0.45\linewidth,
        height=0.45\linewidth,
        xlabel={\tiny Docs},
        ylabel={Score},
        ymin=0, ymax=70,
        ytick={20,40,60},
        tick style={draw=none},
        axis line style={draw=none},
        grid=major,
        grid style={dotted},
        yticklabel style={font=\fontsize{1pt}{2.5pt}\selectfont},
        xticklabel style={font=\fontsize{1pt}{2.5pt}\selectfont, rotate=45, anchor=east},
        title style={font=\tiny},
        xtick=data,
        symbolic x coords={5,10,20,100},
        ylabel style={yshift=-20pt, font=\fontsize{1pt}{2.5pt}\selectfont},
        xlabel style={yshift=6pt, font=\fontsize{1pt}{2.5pt}\selectfont}
    ]

    % LoCoMo subplot
    \nextgroupplot[title={LoCoMo}]
    \addplot+[
        mark=triangle*,
        mark size=0.4pt,
        thick,
        dash pattern=on 1pt off 1pt,
        draw=emcolor,
        mark options={fill=emcolor, draw=emcolor, solid}
    ] table[x=K, y=EM] {\locomo};

    \addplot+[
        mark=triangle*,
        mark size=0.4pt,
        thick,
        dash pattern=on 1pt off 1pt,
        draw=r1color,
        mark options={fill=r1color, draw=r1color, solid}
    ] table[x=K, y=R1] {\locomo};

    \addplot+[
        mark=triangle*,
        mark size=0.4pt,
        thick,
        dash pattern=on 1pt off 1pt,
        draw=r2color,
        mark options={fill=r2color, draw=r2color, solid}
    ] table[x=K, y=R2] {\locomo};

    \addplot+[
        mark=triangle*,
        mark size=0.4pt,
        thick,
        dash pattern=on 1pt off 1pt,
        draw=l1color,
        mark options={fill=l1color, draw=l1color, solid}
    ] table[x=K, y=L1] {\locomo};

    % HotpotQA subplot
    \nextgroupplot[
        title={HotpotQA},
        ylabel={},
        yticklabels={},
        ytick style={draw=none}
    ]
    \addplot+[
        mark=triangle*,
        mark size=0.4pt,
        thick,
        dash pattern=on 1pt off 1pt,
        draw=emcolor,
        mark options={fill=emcolor, draw=emcolor, solid}
    ] table[x=K, y=EM] {\hotpotqa};

    \addplot+[
        mark=triangle*,
        mark size=0.4pt,
        thick,
        dash pattern=on 1pt off 1pt,
        draw=r1color,
        mark options={fill=r1color, draw=r1color, solid}
    ] table[x=K, y=R1] {\hotpotqa};

    \addplot+[
        mark=triangle*,
        mark size=0.4pt,
        thick,
        dash pattern=on 1pt off 1pt,
        draw=r2color,
        mark options={fill=r2color, draw=r2color, solid}
    ] table[x=K, y=R2] {\hotpotqa};

    \addplot+[
        mark=triangle*,
        mark size=0.4pt,
        thick,
        dash pattern=on 1pt off 1pt,
        draw=l1color,
        mark options={fill=l1color, draw=l1color, solid}
    ] table[x=K, y=L1] {\hotpotqa};

    % TwoWiki subplot
    \nextgroupplot[
        title={2Wiki},
        ylabel={},
        yticklabels={},
        ytick style={draw=none}
    ]
    \addplot+[
        mark=triangle*,
        mark size=0.4pt,
        thick,
        dash pattern=on 1pt off 1pt,
        draw=emcolor,
        mark options={fill=emcolor, draw=emcolor, solid}
    ] table[x=K, y=EM] {\twowiki};

    \addplot+[
        mark=triangle*,
        mark size=0.4pt,
        thick,
        dash pattern=on 1pt off 1pt,
        draw=r1color,
        mark options={fill=r1color, draw=r1color, solid}
    ] table[x=K, y=R1] {\twowiki};

    \addplot+[
        mark=triangle*,
        mark size=0.4pt,
        thick,
        dash pattern=on 1pt off 1pt,
        draw=r2color,
        mark options={fill=r2color, draw=r2color, solid}
    ] table[x=K, y=R2] {\twowiki};

    \addplot+[
        mark=triangle*,
        mark size=0.4pt,
        thick,
        dash pattern=on 1pt off 1pt,
        draw=l1color,
        mark options={fill=l1color, draw=l1color, solid}
    ] table[x=K, y=L1] {\twowiki};

    % MuSiQue subplot
    \nextgroupplot[
        title={MuSiQue},
        ylabel={},
        yticklabels={},
        ytick style={draw=none}
    ]
    \addplot+[
        mark=triangle*,
        mark size=0.4pt,
        thick,
        dash pattern=on 1pt off 1pt,
        draw=emcolor,
        mark options={fill=emcolor, draw=emcolor, solid}
    ] table[x=K, y=EM] {\musique};

    \addplot+[
        mark=triangle*,
        mark size=0.4pt,
        thick,
        dash pattern=on 1pt off 1pt,
        draw=r1color,
        mark options={fill=r1color, draw=r1color, solid}
    ] table[x=K, y=R1] {\musique};

    \addplot+[
        mark=triangle*,
        mark size=0.4pt,
        thick,
        dash pattern=on 1pt off 1pt,
        draw=r2color,
        mark options={fill=r2color, draw=r2color, solid}
    ] table[x=K, y=R2] {\musique};

    \addplot+[
        mark=triangle*,
        mark size=0.4pt,
        thick,
        dash pattern=on 1pt off 1pt,
        draw=l1color,
        mark options={fill=l1color, draw=l1color, solid}
    ] table[x=K, y=L1] {\musique};

    \end{groupplot}
\end{tikzpicture}
    \caption{\scriptsize MSMarco-BERT}\label{fig:scores_dense_qwen}
\end{subfigure}
\begin{subfigure}[b]{0.49\textwidth}
    % Define custom colors
\definecolor{emcolor}{rgb}{0.51,0.51,0.71}     % 130,130,180
\definecolor{r1color}{rgb}{0.39,0.75,0.51}     % 100,190,130
\definecolor{r2color}{rgb}{0.90,0.55,0.27}     % 230,140,70
\definecolor{l1color}{rgb}{0.68,0.85,0.90}     % 173,216,230

% Data tables
\pgfplotstableread[row sep=\\,col sep=&]{
K & EM & R1 & R2 & L1 \\
5   & 20.9 & 40.4 & 21.2 & 33.0 \\
10  & 22.1 & 43.7 & 23.2 & 36.0 \\
20  & 22.6 & 46.0 & 24.3 & 38.0 \\
100 & 23.5 & 47.9 & 24.8 & 40.0 \\
}\locomo

\pgfplotstableread[row sep=\\,col sep=&]{
K & EM & R1 & R2 & L1 \\
5   & 33.5 & 42.4 & 24.8 & 43.0 \\
10  & 37.8 & 48.2 & 28.3 & 49.0 \\
20  & 40.8 & 52.0 & 30.1 & 53.0 \\
100 & 42.0 & 54.2 & 31.0 &  \\
}\hotpotqa

\pgfplotstableread[row sep=\\,col sep=&]{
K & EM & R1 & R2 & L1 \\
5   & 27.8 & 31.1 & 25.1 & 28.0 \\
10  & 30.7 & 34.6 & 27.4 & 32.0 \\
20  & 32.0 & 36.7 & 28.3 & 33.0 \\
100 &  &  &  &  \\
}\twowiki

\pgfplotstableread[row sep=\\,col sep=&]{
K & EM & R1 & R2 & L1 \\
5   & 6.7 & 9.7 & 5.5 & 9.0 \\
10  & 8.7 & 12.6 & 7.0 & 11.0 \\
20  & 11.4 & 16.3 & 8.3 & 13.0 \\
100 & 10.6 & 18.0 & 9.0 & 14.0 \\
}\musique

\begin{tikzpicture}
    \begin{groupplot}[
        group style={
            group size=4 by 1,
            horizontal sep=0.05cm,
            vertical sep=0cm,
        },
        width=0.45\linewidth,
        height=0.45\linewidth,
        xlabel={\tiny Docs},
        ylabel={Score},
        ymin=0, ymax=70,
        ytick={20,40,60},
        tick style={draw=none},
        axis line style={draw=none},
        grid=major,
        grid style={dotted},
        yticklabel style={font=\fontsize{1pt}{2.5pt}\selectfont},
        xticklabel style={font=\fontsize{1pt}{2.5pt}\selectfont, rotate=45, anchor=east},
        title style={font=\tiny},
        xtick=data,
        symbolic x coords={5,10,20,100},
        ylabel style={yshift=-20pt, font=\fontsize{1pt}{2.5pt}\selectfont},
        xlabel style={yshift=6pt, font=\fontsize{1pt}{2.5pt}\selectfont}
    ]

    % LoCoMo subplot
    \nextgroupplot[title={LoCoMo}]
    \addplot+[
        mark=triangle*,
        mark size=0.4pt,
        thick,
        dash pattern=on 1pt off 1pt,
        draw=emcolor,
        mark options={fill=emcolor, draw=emcolor, solid}
    ] table[x=K, y=EM] {\locomo};

    \addplot+[
        mark=triangle*,
        mark size=0.4pt,
        thick,
        dash pattern=on 1pt off 1pt,
        draw=r1color,
        mark options={fill=r1color, draw=r1color, solid}
    ] table[x=K, y=R1] {\locomo};

    \addplot+[
        mark=triangle*,
        mark size=0.4pt,
        thick,
        dash pattern=on 1pt off 1pt,
        draw=r2color,
        mark options={fill=r2color, draw=r2color, solid}
    ] table[x=K, y=R2] {\locomo};

    \addplot+[
        mark=triangle*,
        mark size=0.4pt,
        thick,
        dash pattern=on 1pt off 1pt,
        draw=l1color,
        mark options={fill=l1color, draw=l1color, solid}
    ] table[x=K, y=L1] {\locomo};

    % HotpotQA subplot
    \nextgroupplot[
        title={HotpotQA},
        ylabel={},
        yticklabels={},
        ytick style={draw=none}
    ]
    \addplot+[
        mark=triangle*,
        mark size=0.4pt,
        thick,
        dash pattern=on 1pt off 1pt,
        draw=emcolor,
        mark options={fill=emcolor, draw=emcolor, solid}
    ] table[x=K, y=EM] {\hotpotqa};

    \addplot+[
        mark=triangle*,
        mark size=0.4pt,
        thick,
        dash pattern=on 1pt off 1pt,
        draw=r1color,
        mark options={fill=r1color, draw=r1color, solid}
    ] table[x=K, y=R1] {\hotpotqa};

    \addplot+[
        mark=triangle*,
        mark size=0.4pt,
        thick,
        dash pattern=on 1pt off 1pt,
        draw=r2color,
        mark options={fill=r2color, draw=r2color, solid}
    ] table[x=K, y=R2] {\hotpotqa};

    \addplot+[
        mark=triangle*,
        mark size=0.4pt,
        thick,
        dash pattern=on 1pt off 1pt,
        draw=l1color,
        mark options={fill=l1color, draw=l1color, solid}
    ] table[x=K, y=L1] {\hotpotqa};

    % TwoWiki subplot
    \nextgroupplot[
        title={2Wiki},
        ylabel={},
        yticklabels={},
        ytick style={draw=none}
    ]
    \addplot+[
        mark=triangle*,
        mark size=0.4pt,
        thick,
        dash pattern=on 1pt off 1pt,
        draw=emcolor,
        mark options={fill=emcolor, draw=emcolor, solid}
    ] table[x=K, y=EM] {\twowiki};

    \addplot+[
        mark=triangle*,
        mark size=0.4pt,
        thick,
        dash pattern=on 1pt off 1pt,
        draw=r1color,
        mark options={fill=r1color, draw=r1color, solid}
    ] table[x=K, y=R1] {\twowiki};

    \addplot+[
        mark=triangle*,
        mark size=0.4pt,
        thick,
        dash pattern=on 1pt off 1pt,
        draw=r2color,
        mark options={fill=r2color, draw=r2color, solid}
    ] table[x=K, y=R2] {\twowiki};

    \addplot+[
        mark=triangle*,
        mark size=0.4pt,
        thick,
        dash pattern=on 1pt off 1pt,
        draw=l1color,
        mark options={fill=l1color, draw=l1color, solid}
    ] table[x=K, y=L1] {\twowiki};

    % MuSiQue subplot
    \nextgroupplot[
        title={MuSiQue},
        ylabel={},
        yticklabels={},
        ytick style={draw=none}
    ]
    \addplot+[
        mark=triangle*,
        mark size=0.4pt,
        thick,
        dash pattern=on 1pt off 1pt,
        draw=emcolor,
        mark options={fill=emcolor, draw=emcolor, solid}
    ] table[x=K, y=EM] {\musique};

    \addplot+[
        mark=triangle*,
        mark size=0.4pt,
        thick,
        dash pattern=on 1pt off 1pt,
        draw=r1color,
        mark options={fill=r1color, draw=r1color, solid}
    ] table[x=K, y=R1] {\musique};

    \addplot+[
        mark=triangle*,
        mark size=0.4pt,
        thick,
        dash pattern=on 1pt off 1pt,
        draw=r2color,
        mark options={fill=r2color, draw=r2color, solid}
    ] table[x=K, y=R2] {\musique};

    \addplot+[
        mark=triangle*,
        mark size=0.4pt,
        thick,
        dash pattern=on 1pt off 1pt,
        draw=l1color,
        mark options={fill=l1color, draw=l1color, solid}
    ] table[x=K, y=L1] {\musique};

    \end{groupplot}
\end{tikzpicture}
    \caption{\scriptsize BM25}\label{fig:scores_bm25_qwen}
\end{subfigure}

\caption{Performance results of Qwen2.5-14B with different retrievers.}
\label{fig:scores_qwen}
\end{figure}

\noindent We also test o3-mini using $k = 5$. As expected, o3-mini matches or outperforms other models with higher $k$ values, suggesting that reasoning ability plays a key role for MHQA task.

\noindent Finally, we include QA results from HippoRAG \cite{NEURIPS2024_6ddc001d}, which incorporates graph-structured memory. HippoRAG performs especially well on the MuSiQue dataset, likely due to its ability to abstract relationships and reason over structured content.

\noindent Detailed results for all baseline configurations are provided in Table \ref{tab:qa_results}.

\begin{table}[ht]
    \centering
        \adjustbox{max width=1.1\textwidth}{
        \begin{tabular}{lcccccccccccccccccccc}
            \toprule
            & \multicolumn{4}{c}{\textbf{LoCoMo}} & \multicolumn{4}{c}{\textbf{HotpotQA}} & \multicolumn{4}{c}{\textbf{2Wiki}} & \multicolumn{4}{c}{\textbf{MuSiQue}} & \multicolumn{4}{c}{\textbf{Average}} \\
            \cmidrule(lr){2-5} \cmidrule(lr){6-9} \cmidrule(lr){10-13} \cmidrule(lr){14-17} \cmidrule(lr){18-21}
             & EM & $R_1$ & $R_2$ & $L_1$ & EM & $R_1$ & $R_2$ & $L_1$ & EM & $R_1$ & $R_2$ & $L_1$ & EM & $R_1$ & $R_2$ & $L_1$ & EM & $R_1$ & $R_2$ & $L_1$ \\
            \midrule
            $\text{GPT-4o-mini}^1$ & 36.0 & 66.1 & 37.5 & 62.0 & 63.7 & 77.2 & 43.6 & 78.0 & 56.9 & 64.2 & 43.6 & 63.0 & 46.5 & 54.8 & 37.5 & 52.0 & 50.1 & 65.6 & 40.6 & 63.8 \\
            $\text{o3-mini}^1$ & 33.8 & 65.5 & 37.2 & 69.0 & 68.0 & 83.1 & 48.5 & 85.0 & 68.3 & 77.1 & 54.1 & 77.0 & 65.9 & 76.7 & 56.1 & 76.0 & 59.0 & 75.6 & 49.0 & 76.8 \\
            $\text{Qwen2.5-14B}^1$ & 35.0 & 63.3 & 35.0 & 58.0 & 59.0 & 72.9 & 44.1 & 75.0 & 57.3 & 65.6 & 47.1 & 65.0 & 38.3 & 45.7 & 32.2 & 44.0 & 47.4 & 61.9 & 39.6 & 60.5 \\
            GPT-4o-mini IterDRAG\cite{yue2024inferencescalinglongcontextretrieval} & - & - & - & - & - & - & - & - & - & - & - & - & - & - & - & - & - & - & - & - \\
            \midrule
            $\text{GPT-4o-mini }(128k)^2$ & 16.7 & 40.6 & 21.8 & 43.0 & 42.0 & 54.1 & 29.7 & 54.3 & 30.0 & 35.0 & 22.5 & 33.0 & 19.5 & 29.7 & 15.2 & 23.0 & 27.1 & 39.9 & 22.3 & 38.3 \\
            $\text{Qwen2.5-14B }(32k)^2$ & - & - & - & - & - & - & - & - & - & - & - & - & - & - & - & - & - & - & - & - \\
            \midrule
            \multicolumn{21}{c}{\textbf{\small K=5}} \\
            \midrule
            \multicolumn{21}{c}{\textbf{\small GPT-4o-mini}} \\
            \midrule
            BM25 & 23.3 & 43.9 & 23.6 & 39.0 & 36.2 & 44.6 & 25.1 & 45.0 & 26.3 & 28.8 & 21.6 & 26.0 & 7.6 & 10.3 & 5.6 & 9.0 & 23.4 & 31.9 & 19.0 & 29.8  \\
            MSMarco-BERT & 22.5 & 44.3 & 22.8 & 41.0 & 34.7 & 43.1 & 22.7 & 43.0 & 28.1 & 31.0 & 21.6 & 28.0 & 10.7 & 15.2 & 8.8 & 12.0 & 24.0 & 33.4 & 19.0 & 31.0 \\
            ColBERTV2 & 
\textbf{27.5} & \textbf{51.6} & \underline{27.3} & \underline{47.0} & 39.6 & 49.2 & 27.2 & 49.0 & 25.2 & 28.1 & 19.4 & 26.0 & 10.9 & 15.5 & 8.6 & 14.0 & 25.8 & 36.1 & 20.6 & 34.0 \\
            \midrule
            \multicolumn{21}{c}{\textbf{\small o3-mini}} \\
            \midrule
            BM25 & 21.7 & 41.5 & 23.1 & 41.0 & 39.8 & 50.0 & \underline{29.3} & 52.0 & 28.7 & 30.6 & 22.3 & 29.0 & 16.3 & 20.9 & 13.5 & 19.0 & 26.7 & 35.8 & 22.1 & 35.3  \\
            MSMarco-BERT & 20.7 & 43.0 & 22.5 & 44.0 & \underline{42.1} & \underline{52.5} & \underline{29.3} & \underline{54.0} & \textbf{31.6} & \textbf{34.2} & \underline{23.1} & \textbf{33.0} & \underline{19.6} & \underline{26.2} & \underline{16.4} & \underline{23.0} & \underline{28.6} & \underline{39.0} & \underline{22.8} & \underline{38.5} \\
            ColBERTV2 & 
\underline{25.6} & \underline{50.7} & \textbf{27.4} & \textbf{51.0} & \textbf{46.6} & \textbf{57.9} & \textbf{33.5} & \textbf{59.0} & 29.1 & 31.9 & 21.8 & \underline{31.0} & 
\textbf{20.8} & \textbf{27.9} & \textbf{18.0} & \textbf{26.0} & \textbf{30.5} & \textbf{42.1} & \textbf{25.2} & \textbf{41.8} \\
            \midrule
            \multicolumn{21}{c}{\textbf{\small Qwen2.5-14B}} \\
            \midrule
            BM25 & 20.9 & 40.4 & 21.2 & 33.0 & 33.5 & 42.4 & 24.8 & 43.0 & 27.8 & 31.1 & 25.1 & 28.0 & 6.7 & 9.7 & 5.5 & 9.0 & 22.2 & 30.9 & 19.2 & 28.3  \\
            MSMarco-BERT & 21.5 & 42.0 & 21.0 & 34.0 & 32.2 & 40.5 & 22.7 & 41.0 & \underline{29.9} & \underline{33.9} & \textbf{25.5} & \underline{31.0} & 8.4 & 12.6 & 7.2 & 10.0 & 23.0 & 32.3 & 19.1 & 29.0 \\
            ColBERTV2 & 24.9 & 48.6 & 25.4 & 42.0 & 36.9 & 46.8 & 27.2 & 47.0 & 27.2 & 31.3 & 23.5 & 28.0 & 9.7 & 13.9 & 7.5 & 12.0 & 24.7 & 35.2 & 20.9 & 32.3 \\
            \midrule
            \multicolumn{21}{c}{\textbf{\small K=10}} \\
            \midrule
            \multicolumn{21}{c}{\textbf{\small GPT-4o-mini}} \\
            \midrule
            BM25 & 24.8 & 47.1 & 25.6 & 42.0 & \underline{41.2} & \underline{50.7} & 28.7 & \underline{51.0} & 29.8 & 32.6 & 24.2 & 30.0 & 9.0 & 12.8 & 6.9 & 11.0 & 26.2 & 35.8 & 21.4 & 33.5  \\
            MSMarco-BERT & 24.9 & 47.9 & 25.0 & \underline{44.0} & 37.6 & 46.7 & 25.0 & 47.0 & \underline{31.0} & 34.2 & 24.2 & \underline{32.0} & 
\textbf{13.9} & \textbf{18.7} & \textbf{10.8} & \textbf{16.0} & \underline{26.9} & 36.9 & 21.3 & \underline{34.8} \\
            ColBERTV2 & 
\textbf{27.3} & \textbf{53.0} & \textbf{28.3} & \textbf{47.0} & \textbf{43.2} & \textbf{53.4} & \textbf{29.7} & \textbf{53.0} & 27.9 & 31.2 & 21.9 & 29.0 & \underline{13.1} & \underline{17.4} & 
\underline{9.9} & \underline{15.0} & \textbf{27.9} & \textbf{38.8} & \textbf{22.5} & \textbf{36.0} \\
            \midrule
            \multicolumn{21}{c}{\textbf{\small Qwen2.5-14B}} \\
            \midrule
            BM25 & 22.1 & 43.7 & 23.2 & 36.0 & 37.8 & 48.2 & 28.3 & 49.0 & 30.7 & \underline{34.6} & \underline{27.4} & \underline{32.0} & 8.7 & 12.6 & 7.0 & 11.0 & 24.8 & 34.8 & 21.5 & 32.0  \\
            MSMarco-BERT & 22.4 & 44.7 & 21.8 & 37.0 & 34.8 & 44.1 & 24.8 & 44.0 & \textbf{32.4} & \textbf{37.0} & \textbf{27.9} & \textbf{34.0} & 12.2 & 17.2 & 
9.6 & 14.0 & 25.5 & 35.8 & 21.0 & 32.3 \\
            ColBERTV2 & 
\underline{25.4} & \underline{50.0} & \underline{25.8} & 42.0 & 39.8 & 50.5 & \underline{29.2} & \underline{51.0} & 29.8 & 34.5 & 25.7 & 31.0 & 11.7 & 16.6 & 9.0 & 14.0 & 26.7 & \underline{37.9} & \underline{22.4} & 34.5 \\
            \midrule
            \multicolumn{21}{c}{\textbf{\small K=20}} \\
            \midrule
            \multicolumn{21}{c}{\textbf{\small GPT-4o-mini}} \\
            \midrule
            BM25 & 25.4 & 49.2 & 26.0 & 45.0 & \underline{45.1} & \underline{55.3} & \underline{31.3} & \underline{56.0} & 30.8 & 34.0 & 25.1 & 31.0 & 10.9 & 15.1 & 8.3 & 13.0 & 28.1 & 38.4 & 22.7 & 36.3  \\
            MSMarco-BERT & \underline{25.6} & 50.2 & 26.2 & \underline{47.0} & 40.0 & 49.5 & 26.6 & 50.0 & 32.7 & 36.1 & 25.8 & 33.0 & \textbf{15.5} & \textbf{20.6} & \textbf{12.4} & \textbf{18.0} & \underline{28.5} & 39.1 & 22.8 & \underline{37.0} \\
            ColBERTV2 & 
\textbf{28.4} & \textbf{55.1} & \textbf{30.0} &  \textbf{51.0} & \textbf{46.0} & \textbf{56.7} & \textbf{31.5} & \textbf{57.0} & 31.5 & 35.2 & 24.9 & \underline{35.0} & \underline{14.4} & 19.0 & 11.1 & \underline{17.0} & \textbf{30.1} & \textbf{41.5} & \textbf{24.4} & \textbf{40.0} \\
            \midrule
            \multicolumn{21}{c}{\textbf{\small Qwen2.5-14B}} \\
            \midrule
            BM25 & 22.6 & 46.0 & 24.3 & 38.0 & 40.8 & 52.0 & 30.1 & 53.0 & 32.0 & 36.7 & 28.3 & 33.0 & 11.4 & 16.3 & 8.3 & 13.0 & 26.7 & 37.8 & 22.8 & 34.3  \\
            MSMarco-BERT & 24.6 & 48.4 & 24.5 & 41.0 & 36.7 & 46.6 & 26.2 & 47.0 & \textbf{34.0} & \textbf{38.9} & \textbf{29.2} &  \textbf{36.0} & 13.9 & \underline{20.4} & \underline{11.2} & \underline{17.0} & 27.3 & 38.6 & 22.8 & 35.3 \\
            ColBERTV2 & \underline{25.6} & \underline{52.2} & \underline{27.0} & 43.0 & 41.9 & 53.3 & 30.8 & 54.0 & \underline{33.1} & \underline{38.3} & \underline{28.6} & \underline{35.0} & 13.0 & 19.0 & 10.3 & 16.0 & 28.4 & \underline{40.7} & \underline{24.2} & \underline{37.0} \\
            \midrule
            \multicolumn{21}{c}{\textbf{\small K=100}} \\
            \midrule
            \multicolumn{21}{c}{\textbf{\small GPT-4o-mini}} \\
            \midrule
            BM25 & \underline{27.5} & 53.4 & \underline{29.2} & 51.0 & \underline{46.1} & \underline{57.6} & \underline{31.8} & \underline{59.0} & 33.0 & 36.7 & 26.3 & 34.0 & 12.4 & 17.1 & 9.9 & 15.0 & 29.8 & 41.2 & 24.3 & 39.8  \\
            MSMarco-BERT & 26.7 & \underline{53.9} & 28.5 & \underline{52.0} & 42.6 & 52.9 & 28.5 & 54.0 & \textbf{34.9} & 38.8 & 27.3 & \underline{36.0} & 
\textbf{15.5} & \underline{21.0} & \textbf{13.1} & \textbf{19.0} & \underline{29.9} & 41.7 & 24.4 & \underline{40.3} \\
            ColBERTV2 & 
\textbf{29.2} & \textbf{56.7} & \textbf{31.2} & \textbf{54.0} & \textbf{47.5} & \textbf{59.2} & \textbf{32.7} & \textbf{60.7} & 33.4 & 37.5 & 26.1 & 35.0 & \underline{15.2} & 20.7 & \underline{11.9} & \textbf{19.0} & \textbf{31.3} & \textbf{43.5} & \textbf{25.5} & \textbf{42.2} \\
            \midrule
            \multicolumn{21}{c}{\textbf{\small Qwen2.5-14B}} \\
            \midrule
            BM25 & 23.5 & 47.9 & 24.8 & 40.0 & 42.0 & 54.2 & 31.0 & 55.0 & 33.3 & 38.8 & 28.4 & 35.0 & 10.6 & 18.0 & 9.0 & 14.0 & 27.3 & 39.7 & 23.3 & 36.0  \\
            MSMarco-BERT & 24.8 & 50.0 & 25.2 & 41.0 & 39.2 & 50.7 & \underline{28.3} & 52.0 & \underline{34.7} & \underline{40.3} & \textbf{29.3} & \textbf{37.0} & 14.2 & \textbf{21.8} & 11.8 & \underline{18.0} & 28.2 & 40.7 & 23.7 & 37.0 \\
            ColBERTV2 & 26.6 & 53.0 & 27.0 & 45.0 & 42.5 & 55.2 & 31.2 & 56.0 & \underline{34.7} & \textbf{40.7} & \textbf{29.3} & \textbf{37.0} & 12.5 & 19.7 & 10.3 & 16.0 & 29.1 & \underline{42.2} & \underline{24.5} & 38.5 \\
            \midrule
            HippoRAG (contriever) & - & - & - & - & 34.3 & 48.0 & 25.7 & 53.0 & 14.7 & 23.0 & 12.5 & 23.0 & 12.5 & 20.2 & 11.9 & 19.0 & - & - & - & - \\
            \bottomrule
            \multicolumn{16}{l}{\rule{0pt}{1.25em}
            \text{$^1$ \footnotesize Only the supporting passages are passed to the LLM for answering the questions.}
            } \\
            \multicolumn{16}{l}{\rule{0pt}{0em}
            \text{$^2$ \footnotesize The entire corpus is passed to the LLM for answering the question. The corpus may be truncated.}
            }
        \end{tabular}
        }
    \caption{\small Question-Answering performance.}
    \label{tab:qa_results}
\end{table}
