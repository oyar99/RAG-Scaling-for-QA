\usepackage{listings}
\usepackage{color}
\definecolor{darkgray}{rgb}{0.35,0.35,0.35}
\definecolor{navgray}{rgb}{0.85,0.85,0.85}

\renewcommand{\lstlistingname}{Programa}
\renewcommand{\lstlistlistingname}{Índice de Programas}


% Basic definition of a listing
\lstset{ 
basicstyle= \fontfamily{pcr}\selectfont\scriptsize,       	% the size of the fonts that are used for the code
stringstyle=\color{darkgray}\slshape\bfseries,
keywordstyle=\bfseries, 		% style of normal keywords
keywordstyle={[2]\underbar}, 	% style of second set of keywords
identifierstyle=,				% nothing happens 
showtabs=false,                 		% show tabs within strings adding particular underscores
tabsize=2,	                			% sets default tabsize to 2 spaces
showspaces=false,               	% show spaces adding particular underscores
showstringspaces=false,		% show spaces in strings
morestring=[b]",
backgroundcolor=\color{white},  	% choose the background color. You must add \usepackage{color}
numbers=left,                   		% where to put the line-numbers
numberstyle=\tiny, %scriptsize,      	% the size of the fonts that are used for the line-numbers
stepnumber=1,                   	% the step between two line-numbers.
numbersep=3pt,                  	% how far the line-numbers are from the code
frame=b, %single,	                		% adds a frame around the code
captionpos=t, %b                   		% sets the caption-position to bottom
xleftmargin=10pt,
xrightmargin=6pt,
framexleftmargin=10pt,
framexrightmargin=5pt,
framexbottommargin=3pt
}

% Definition of CCL
\lstdefinelanguage{CCL}{
morekeywords={assembly,new,on,Init, global,load,Instance,ActionLink,ModelInstanceCreation}, % Keywords
morekeywords={[2]XTM,miniBPMN}, % Additional keywords
morecomment=[l]{//},  % Configuration of comments
morecomment=[s]{/*}{*/}
}

% Definition of Cumbia's XML
\lstdefinelanguage{CumbiaXML}{
language=XML,                % choose the language of the code
morekeywords={}, % Keywords in these XMLS
morekeywords={[2]extended_type,new_entity,new_state_machine,state_machine_extensions, add_intermediate_state,additional_state,additional_transition,source_event,before_event,after_event, new_relation,new_role,role_detail,new_event,action,add_actions,add_transition,actions,add_state,transition,incoming_transitions,outgoing_transitions,relation,metamodel,state_machine_reference,type_name,role,type,runtime,memory,data,metamodel_extensions,definition,model_structure,elements,connections,connection,element,state_machine,state,event}, % Elements
keywordstyle={[2]\bfseries}, 	% style of second set of keywords
morekeywords={[3]extends,name,entityClass,file,transitionName,enter_event,exit_event,source_name,source_state,event_name,cardinality,targetTypeName,interface,type,relationType,description,class,successor,initial_state,location,statemachine,abstract,version,root,modelName,relationName,sourceElement,targetElement,typeName}, % attributes
keywordstyle={[3]}, 	% style of third set of keywords
breaklines=true,                	% sets automatic line breaking
breakatwhitespace=false,	% sets if automatic breaks should only happen at whitespace
escapeinside={\%*}{*)}	% configuration of comments
}


% List of listings
\newcommand{\listListingsName}{List of Listings}
\newlistof[chapter]{listings}{lst}{\listListingsName}
	
% DEFINE THE ``xmlcode'' ENVIRONMENT
\lstnewenvironment{xmlcode}[2]
{
	\refstepcounter{listings}
	\label{lst:#2}	
	\addcontentsline{lst}{listings}{\protect\numberline {\thelistings} #1}
	\lstset{language=CumbiaXML,breaklines=true,breakatwhitespace=false,caption=#1,float=phtb}
}
{
}

% DEFINE THE ``hxmlcode'' ENVIRONMENT
\lstnewenvironment{hxmlcode}[2]
{
	\refstepcounter{listings}
	\label{lst:#2}	
	\addcontentsline{lst}{listings}{\protect\numberline {\thelistings} #1}
	\lstset{language=CumbiaXML,breaklines=true,breakatwhitespace=false,caption=#1,float=h!t}
}
{
}

% DEFINE THE ``cclcode'' ENVIRONMENT
\lstnewenvironment{cclcode}[2]
{
	\refstepcounter{listings}
	\label{lst:#2}	
	\addcontentsline{lst}{listings}{\protect\numberline {\thelistings} #1}
	\lstset{language=CCL,caption=#1,breaklines=true,breakatwhitespace=false,float=phtb}
}
{
}

% DEFINE THE ``javacode'' ENVIRONMENT
\lstnewenvironment{javacode}[2]
{
	\refstepcounter{listings}
	\label{lst:#2}	
	\addcontentsline{lst}{listings}{\protect\numberline {\thelistings} #1}
	\lstset{language=Java,caption=#1,breaklines=true,breakatwhitespace=false,float=phtb}
}
{
}


% Definition of NavLang
\lstdefinelanguage{NavLang}{
backgroundcolor=\color{navgray},
basicstyle= \fontfamily{pcr}\selectfont\scriptsize,
morekeywords={\#,root,this,self,nil,isOfType}, % Keywords
morecomment=[l]{//},  % Configuration of comments
morecomment=[s]{/*}{*/},
otherkeywords={::,&&,isNotNil\?,<!,>,->,\#},
frame=none,
numbers=none,                   		% remove line-numbers
linewidth=0.95\columnwidth,
xleftmargin=15pt
}

% DEFINE THE ``navcode'' ENVIRONMENT
\lstnewenvironment{navcode}
{
	\lstset{language=NavLang,breaklines=true,breakatwhitespace=false}
}
{
}

\newcommand{\nav}[1]
{\colorbox{navgray}{\lstinline[language=NavLang,breaklines=false,breakatwhitespace=false,basicstyle=\normalsize\ttfamily]$#1$}}
